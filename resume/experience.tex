\cvsection{Experience}

\vspace{-0.1cm}

\begin{cventries}
\vspace{-0.01cm}
  \cventry
    {} % Empty position
    {\textbf{Wisconsin IceCube Particle Astrophysics Center \& The IceCube Neutrino Observatory}} % Project
    {\textcolor{gray}{11/17 - present}} % Empty location
    {} % Empty date
    {
      \begin{cvitems} % Description(s) bullet points
      \vspace{-0.4cm}
        \item {Solve problems in modern astrophysics by building low-latency data processing and statistical inference pipelines deployed at the geographic South Pole. Recipient of ``IceCube Impact Award'' for lasting impacts on the collaboration from robust code development and documentation.}
      \end{cvitems}
    }
\end{cventries}

\vspace{-0.15cm}

% \begin{subs}
% \cvsubsection{\normalsize{\hspace{0.2in}Projects}}
\vspace{-0.3cm}
\begin{cventries}
  \cventry
    {\hspace{0.25in}\textbf{Discovering the Sources of High-Energy Astrophysical Neutrinos}}
    {}
    {}
    {11/17 - present}
    {
    \begin{nestcvitems}
      \vspace{-0.1cm}
    %   \textbf{Skills Summary: \textcolor{mydarkblue}{Data Analysis},
    %   \textcolor{mydarkblue}{Scientific Python}, \textcolor{mydarkblue}{Statistical Inference}, \textcolor{mydarkblue}{Pipelining}}
    %   \item {\textcolor{white}{..}}
    %   \vspace{-14pt}
      \item {Performed likelihood analyses on large particle physics and astronomical datasets using Python and scientific computing libraries such as \texttt{NumPy}, \texttt{SciPy}, \texttt{scikit-learn}, \texttt{pandas}, etc.}
      \item {Constructed a low-latency pipeline that extracts insights from 200~GB of daily raw data. Used this pipeline to communicate results publicly and inform astronomical observing strategies over one hundred times.}
      \item {Saved experiment tens of thousands of computation hours by optimizing multiple likelihood algorithms that analyze astrophysical neutrino alert data transmitted via satellite from the South Pole.}
    %   \item{Major contributor to 7 papers published in major physics and computing journals.}
      \item{Core developer and active maintainer of two open-source Python packages that are widely used in my scientific discipline~-- \href{https://github.com/icecube/FIRESONG}{\faGithub\acvHeaderIconSep\texttt{FIRESONG}} and \href{https://github.com/icecube/TauRunner}{\faGithub\acvHeaderIconSep\texttt{TauRunner}}. Contributions include optimizing the algorithm run-times and designing the testing suite using pytest and GitHub Actions.}
      \item {Collaborate effectively with groups of scientists, post-doctoral fellows, and graduate students across the globe. Mentored multiple graduate and undergraduate physics and computer science students.}
    \end{nestcvitems}
    }
\vspace{-0.1cm}   
\cventry
    {\hspace{0.25in}\textbf{Searching for neutrino emission from explosive optical transients}}
    {}
    {}
    {1/19 - present}
    {
    \begin{nestcvitems}
      \vspace{-0.1cm}
    %   \textbf{Skills Summary: \textcolor{mydarkblue}{Machine Learning},
    %   \textcolor{mydarkblue}{Scientific Python}, \textcolor{mydarkblue}{Data Visualization}, \textcolor{mydarkblue}{High-Throughput Computing}}
    %   \item {\textcolor{white}{..}}
    %   \vspace{-14pt}
      \item {Developed a method to characterize uncertainties for low-energy neutrino events by training a random forest using \texttt{scikit-learn} on neutrino simulation and then applying this to neutrino data, which is now used widely within my scientific collaboration.}
      \item {Performed a likelihood-based hypothesis test to execute the first search for neutrino emission from Galactic Novae.}
    \end{nestcvitems}
    }
\vspace{-0.1cm}
  \cventry
    {\hspace{0.25in}\textbf{Converting Cell Phones into Particle Detectors}}
    {}
    {}
    {5/18 - present}
    {
    \begin{nestcvitems}
      \vspace{-0.1cm}
    %   \textbf{Skills Summary: \textcolor{mydarkblue}{Machine Learning},  
    %   \textcolor{mydarkblue}{Monte Carlo Methods}, \textcolor{mydarkblue}{Data Visualization}}
    %   \item {\textcolor{white}{..}}
    %   \vspace{-14pt}
      \item {Maintain \href{https://wipac.wisc.edu/deco/home}{cell phone app} that detects cosmic rays with users on all 7 continents.}
      \item {Re-trained \texttt{tensorflow}-based convolutional neural network to discriminate between varieties of cosmic rays in cell phone images with a classification accuracy of over 95\%.}
      \item {Perform image classification with convolutional neural network for images transmitted from a global array of cell phones.}
    \end{nestcvitems}
    }
    
\end{cventries}

% \end{subs}


\begin{cventries}
\vspace{-0.15cm}
  \cventry
    {} % Empty position
    {\textbf{Astrobites} -- \textit{Science Writer}} % Project
    {\textcolor{gray}{1/20 - 1/22}} % Empty location
    {} % Empty date
    {
      \begin{cvitems} % Description(s) bullet points
    %   \vspace{-0.5cm}
    %   \textbf{Skills Summary: \textcolor{mydarkblue}{Science Communication},  
    %   \textcolor{mydarkblue}{Technical Writing}, \textcolor{mydarkblue}{Remote Collaboration}}
    %   \item {\textcolor{white}{..}}
        \vspace{-0.4cm}
        \item {Disseminate complex cutting-edge astrophysics research to the public through collaborative international scientific blog with over 0.5M yearly views. For select articles, see \href{https://astrobites.org/author/apizzuto/}{\textcolor{cyan}{https://astrobites.org/author/apizzuto/}}.}
        \item {Provide press coverage for select scientific conferences.}
      \end{cvitems}
    }
\end{cventries}

